\usepackage[references]{latex/agda}
\usepackage{amsfonts}
\usepackage{amsmath,amssymb,mathtools}
\usepackage{amsthm}
\usepackage{booktabs}
\usepackage{caption}  
\usepackage{cite}
\usepackage{dirtree}
\usepackage{fontspec}
\usepackage{graphicx}
\usepackage{hyperref}
\usepackage{listings}
\usepackage{lmodern}
\usepackage{newunicodechar}
\usepackage{stmaryrd}
\usepackage{tikz-cd}
\usepackage{tikz}
\usepackage{ucharclasses}
\usepackage{unicode-math}
\usepackage{xcolor}
\usepackage{xspace}
% Make the diacritic dot symbol invisible to distinguish symbols
\newunicodechar{̣}{}

\usepackage{fontspec}
\usefonttheme{professionalfonts}
\setsansfont{Liberation Sans}
\setmainfont{Liberation Sans}
\setmonofont{Liberation Sans}

%% \usetheme{Madrid}
%% \usecolortheme{dolphin}
%% \setbeamertemplate{navigation symbols}{}
%% \setbeamertemplate{caption}[numbered]
%% \setbeamertemplate{itemize items}[default]
%% \setbeamertemplate{enumerate items}[default]
%% \setbeamercolor{item projected}{bg=green!70!black,fg=blue}
%% \setbeamercolor{enumerate subitem}{fg=red!80!black}
%% \setbeamertemplate{enumerate items}[default]
%% \usefonttheme{serif}

%% \setmainfont{DejaVu Serif}
%% \setsansfont{Liberation Sans}
%% \setmonofont{JuliaMono}
%% \let\oldttfamily\ttfamily
%% \renewcommand{\ttfamily}{\footnotesize\oldttfamily}

% Agda code style
\renewcommand{\AgdaCodeStyle}{%
	\setmainfont{JuliaMono}
	\setsansfont{JuliaMono}
	\setmonofont{JuliaMono}
	\footnotesize % Make the code smaller to fit within width.
}
\setlength{\mathindent}{0pt}

% TikZ
\usetikzlibrary{positioning,chains,fit,shapes,calc,arrows.meta,decorations.markings}
\tikzset{
dotnode/.style = {circle, fill, inner sep=0.5mm},
thickedge/.style  = {thick},
largestep/.style = {
ultra thick,
-{>[length=8mm, width=8mm]},
shorten >=1cm,
shorten <=1cm},
pivot/.style={circle, fill=white, inner sep=0.5mm, draw=red},
midarrow/.style={
		postaction={
				decorate,
				decoration={
						markings,
						mark=at position 0.5 with {\arrow{>}}
					}
			}
	}
}

\newcommand{\dotrow}[3]{% end, label, labelpos
	\begin{scope}[start chain=going right]
		\foreach \i in {0,...,#1}
		\node[on chain, dotnode, label=#3:{\i}] (#2\i) {};
	\end{scope}
}

% Pandoc expects this to be defined.
\providecommand{\tightlist}{%
	\setlength{\itemsep}{0pt}\setlength{\parskip}{0pt}
}

\definecolor{darkgreen}{HTML}{023020}
\definecolor{darkblue}{HTML}{00008b}


\newcommand{\TraceStepZero}{%
\begin{tikzpicture}[thickedge, node distance=7mm]
  \dotrow{3}{b}{above}
  \begin{scope}[yshift=-2cm]
    \dotrow{2}{a}{below}
  \end{scope}
  \draw[midarrow] (a0) -- (b2);
  \draw[midarrow] (a1) -- (b0);
  \draw[midarrow] (a2) -- (b3);
\end{tikzpicture}%
}

\newcommand{\TraceStepOne}{%
  \begin{tikzpicture}[thickedge, node distance=7mm]
    \dotrow{3}{b}{above}
    \begin{scope}[yshift=-2cm]
      \dotrow{2}{a}{below}
    \end{scope}
    \draw[midarrow] (a0) -- (b2);
    \draw[midarrow] (a1) -- (b0);
    \draw[midarrow] (a2) -- (b3);
    \draw[midarrow] (0,0)
      arc [start angle=0, end angle=180, radius=5mm]
      -- (-1,-2)
      arc [start angle=180, end angle=360, radius=5mm];
  \end{tikzpicture}%
}

\newcommand{\TraceStepTwo}{%
  \begin{tikzpicture}[thickedge, node distance=7mm]
    \dotrow{2}{b}{above}
    \begin{scope}[yshift=-2cm]
      \dotrow{1}{a}{below}
    \end{scope}
    \node[dotnode] (bn1) at (-1,0) {};
    \node[dotnode] (an1) at (-1,-2) {};
    \draw[midarrow] (an1) -- (b1);
    \draw[midarrow] (a0) -- (bn1);
    \draw[midarrow] (a1) -- (b2);
    \draw[midarrow] (-1,0)
      arc [start angle=0, end angle=180, radius=5mm]
      -- (-2,-2)
      arc [start angle=180, end angle=360, radius=5mm];
  \end{tikzpicture}%
}

\newcommand{\TraceStepThree}{%
  \begin{tikzpicture}[thickedge, node distance=7mm]
    \dotrow{2}{b}{above}
    \begin{scope}[yshift=-2cm]
      \dotrow{1}{a}{below}
    \end{scope}
    \draw[midarrow] (a0) -- (b1);
    \draw[midarrow] (a1) -- (b2);
  \end{tikzpicture}
}
