\documentclass[aspectratio=169,16pt]{beamer}
\usepackage[references]{latex/agda}
\usepackage{amsfonts}
\usepackage{amsmath,amssymb,mathtools}
\usepackage{amsthm}
\usepackage{booktabs}
\usepackage{caption}  
\usepackage{cite}
\usepackage{dirtree}
\usepackage{fontspec}
\usepackage{graphicx}
\usepackage{hyperref}
\usepackage{listings}
\usepackage{lmodern}
\usepackage{newunicodechar}
\usepackage{stmaryrd}
\usepackage{tikz-cd}
\usepackage{tikz}
\usepackage{ucharclasses}
\usepackage{unicode-math}
\usepackage{xcolor}
\usepackage{xspace}
% Make the diacritic dot symbol invisible to distinguish symbols
\newunicodechar{̣}{}

\usepackage{fontspec}
\usefonttheme{professionalfonts}
\setsansfont{Liberation Sans}
\setmainfont{Liberation Sans}
\setmonofont{Liberation Sans}

%% \usetheme{Madrid}
%% \usecolortheme{dolphin}
%% \setbeamertemplate{navigation symbols}{}
%% \setbeamertemplate{caption}[numbered]
%% \setbeamertemplate{itemize items}[default]
%% \setbeamertemplate{enumerate items}[default]
%% \setbeamercolor{item projected}{bg=green!70!black,fg=blue}
%% \setbeamercolor{enumerate subitem}{fg=red!80!black}
%% \setbeamertemplate{enumerate items}[default]
%% \usefonttheme{serif}

%% \setmainfont{DejaVu Serif}
%% \setsansfont{Liberation Sans}
%% \setmonofont{JuliaMono}
%% \let\oldttfamily\ttfamily
%% \renewcommand{\ttfamily}{\footnotesize\oldttfamily}

% Agda code style
\renewcommand{\AgdaCodeStyle}{%
	\setmainfont{JuliaMono}
	\setsansfont{JuliaMono}
	\setmonofont{JuliaMono}
	\footnotesize % Make the code smaller to fit within width.
}
\setlength{\mathindent}{0pt}

% TikZ
\usetikzlibrary{positioning,chains,fit,shapes,calc,arrows.meta,decorations.markings}
\tikzset{
dotnode/.style = {circle, fill, inner sep=0.5mm},
thickedge/.style  = {thick},
largestep/.style = {
ultra thick,
-{>[length=8mm, width=8mm]},
shorten >=1cm,
shorten <=1cm},
pivot/.style={circle, fill=white, inner sep=0.5mm, draw=red},
midarrow/.style={
		postaction={
				decorate,
				decoration={
						markings,
						mark=at position 0.5 with {\arrow{>}}
					}
			}
	}
}

\newcommand{\dotrow}[3]{% end, label, labelpos
	\begin{scope}[start chain=going right]
		\foreach \i in {0,...,#1}
		\node[on chain, dotnode, label=#3:{\i}] (#2\i) {};
	\end{scope}
}

% Pandoc expects this to be defined.
\providecommand{\tightlist}{%
	\setlength{\itemsep}{0pt}\setlength{\parskip}{0pt}
}

\definecolor{darkgreen}{HTML}{023020}
\definecolor{darkblue}{HTML}{00008b}


\newcommand{\TraceStepZero}{%
\begin{tikzpicture}[thickedge, node distance=7mm]
  \dotrow{3}{b}{above}
  \begin{scope}[yshift=-2cm]
    \dotrow{2}{a}{below}
  \end{scope}
  \draw[midarrow] (a0) -- (b2);
  \draw[midarrow] (a1) -- (b0);
  \draw[midarrow] (a2) -- (b3);
\end{tikzpicture}%
}

\newcommand{\TraceStepOne}{%
  \begin{tikzpicture}[thickedge, node distance=7mm]
    \dotrow{3}{b}{above}
    \begin{scope}[yshift=-2cm]
      \dotrow{2}{a}{below}
    \end{scope}
    \draw[midarrow] (a0) -- (b2);
    \draw[midarrow] (a1) -- (b0);
    \draw[midarrow] (a2) -- (b3);
    \draw[midarrow] (0,0)
      arc [start angle=0, end angle=180, radius=5mm]
      -- (-1,-2)
      arc [start angle=180, end angle=360, radius=5mm];
  \end{tikzpicture}%
}

\newcommand{\TraceStepTwo}{%
  \begin{tikzpicture}[thickedge, node distance=7mm]
    \dotrow{2}{b}{above}
    \begin{scope}[yshift=-2cm]
      \dotrow{1}{a}{below}
    \end{scope}
    \node[dotnode] (bn1) at (-1,0) {};
    \node[dotnode] (an1) at (-1,-2) {};
    \draw[midarrow] (an1) -- (b1);
    \draw[midarrow] (a0) -- (bn1);
    \draw[midarrow] (a1) -- (b2);
    \draw[midarrow] (-1,0)
      arc [start angle=0, end angle=180, radius=5mm]
      -- (-2,-2)
      arc [start angle=180, end angle=360, radius=5mm];
  \end{tikzpicture}%
}

\newcommand{\TraceStepThree}{%
  \begin{tikzpicture}[thickedge, node distance=7mm]
    \dotrow{2}{b}{above}
    \begin{scope}[yshift=-2cm]
      \dotrow{1}{a}{below}
    \end{scope}
    \draw[midarrow] (a0) -- (b1);
    \draw[midarrow] (a1) -- (b2);
  \end{tikzpicture}
}


\title{Toward a Formalization of 'Virtual Sets'}
\author{Christina O'Donnell\\ Under supervision of Thorsten Altenkirch}
\subtitle{School of Computer Science, University of Nottingham}
\date{September 2025}

\begin{document}

\begin{frame}
  \titlepage
\end{frame}

\begin{frame}{Overview}
  The aim of this project was to formally construct a structure called
  a 'Traced Monoidal Category' around the notion of injective
  functions on finite sets. And implementing the \textrm{Int}
  construction to explore a notion of 'negative sets'.

\begin{center}
  \begin{tikzpicture}[thickedge, node distance=7mm]
    \dotrow{3}{b}{above}
    \begin{scope}[yshift=-2cm]
      \dotrow{2}{a}{below}
    \end{scope}
    \draw[midarrow] (a0) -- (b2);
    \draw[midarrow] (a1) -- (b0);
    \draw[midarrow] (a2) -- (b3);
    \draw[midarrow] (0,0)
      arc [start angle=0, end angle=180, radius=5mm]
      -- (-1,-2)
      arc [start angle=180, end angle=360, radius=5mm];
  \end{tikzpicture}%
  \captionof{figure}{$(2\ 0\ 3)$}
\end{center}
\end{frame}

\begin{frame}{Objectives}
  \begin{itemize}
  \item Construct the category of finite sets and injective in Agda
  \item $(\oplus)$ Define a tensor produt and prove coherence
  \item $(\ominus)$ Define a trace operation and prove coherence
  \item Formalize the \textbf{Int}-construction to produce virutal sets
  \end{itemize}
\end{frame}

\begin{frame}{Methodology}
  \begin{itemize}
    \item Written in Cubical Agda, a proof assistant and
      implementation of homotopy type theory.
    \item Two different representations of injective functions were
      used: depdendent sum representation, and inductive representation.
    \item Work was performed between June and September, and involved
      the production of 5.6k SLoC accross 74 files with 340
      definitions.
  \end{itemize}
\end{frame}

\begin{frame}{Category Construction}
  \begin{itemize}
    \item A category is an abstract structure with many examples.
    \item Many examples come from algebraic objects and
      structure preserving maps called \textit{morphisms}.
    \item We use the following mapping:
    \begin{itemize}
    \item \textbf{objects} are natural numbers ($\mathbb{N}$) representing each
      finite set size.
    \item \textbf{morphisms} are injective maps between finite sets, defined
      inductively.
    \item \textbf{composition} comes from joining two function graphs in a way
      that preserves application.
    \item \textbf{identity} is just the graph that maps all elements to themselves.
    \item \textbf{associativity} and \textbf{left/right-unit} laws also hold.
    \end{itemize}
  \end{itemize}
\end{frame}

\begin{frame}{Tensor Product ($\oplus$)}
  Components of the tensor product:
  \begin{itemize}
  \item object level: arithmetic +
  \item morphism level: split \to apply \to join (see next slide)
  \item unit object: $𝟘 = 0$
  \item preserves identity: $\mathit{id} \oplus \mathit{id} = \mathit{id}$
  \item preserves composition: $(f \circ f') \oplus (g \circ g') = (f \oplus g) \circ (f' \oplus g')$
  \item associator: $\alpha_{A,B,C} : A \oplus (B \oplus C) \cong (A \oplus B) \oplus C$. 
  \item left-unitor: $\eta_{A} : 𝟘 \oplus A = A$
  \item right-unitor: $\rho_{A} : A \oplus 𝟘 = A$
  \item coherence laws: Show result of composition is indepdendent of order
  \end{itemize}
\end{frame}


\begin{frame}{Tensor Product ($\oplus$): Example}
\begin{center}
\begin{tikzpicture}[rectangle, thickedge, node distance=7mm]
  \begin{scope}[shift={(-0.5, -1)}]
    \dotrow{6}{f}{above}
  \end{scope}
  \begin{scope}[shift={(-1, -2)}]
    \dotrow{2}{d}{right}
  \end{scope}
  \begin{scope}[shift={(2.5, -2)}]
    \dotrow{3}{e}{right}
  \end{scope}
  \begin{scope}[shift={(-0.5, -4)}]
    \dotrow{1}{b}{right}
  \end{scope}
  \begin{scope}[shift={(2.5, -4)}]
    \dotrow{3}{c}{right}
  \end{scope}
  \begin{scope}[shift={(0, -5)}]
    \dotrow{5}{a}{below}
  \end{scope}
  \draw (a0) -- (b0);
  \draw (a1) -- (b1);
  \draw (a2) -- (c0);
  \draw (a3) -- (c1);
  \draw (a4) -- (c2);
  \draw (a5) -- (c3);
  \draw[color=green] (b0) -- (d2);
  \draw[color=green] (b1) -- (d0);
  \draw[color=blue] (c0) -- (e3);
  \draw[color=blue] (c1) -- (e1);
  \draw[color=blue] (c2) -- (e0);
  \draw[color=blue] (c3) -- (e2);
  \draw (d0) -- (f0);
  \draw (d1) -- (f1);
  \draw (d2) -- (f2);
  \draw (e0) -- (f3);
  \draw (e1) -- (f4);
  \draw (e2) -- (f5);
  \draw (e3) -- (f6);
\end{tikzpicture}
\captionof{figure}{$(2\ 0) \oplus (3\ 1 \ 0\ 2) = (2\ 0\ 6\ 4\ 3\ 5)$}
\end{center}
\end{frame}

\begin{frame}{Trace ($\ominus$): By example}
  \Large We will construct $\mathrm{tr}_1((2\ 0\ 3)$: \\
  \centering
  \resizebox{0.5\linewidth}{!}{\TraceStepZero}
\end{frame}

\begin{frame}{Trace ($\ominus$): Example: Step 1 Add Loop}
  \begin{columns}[T,onlytextwidth]
    \column{0.48\textwidth}
  \centering
  \resizebox{0.5\linewidth}{!}{\TraceStepZero}

  \column{0.04\textwidth}
  \centering \Large $\Longrightarrow$

  \column{0.48\textwidth}
  \centering
  \resizebox{0.5\linewidth}{!}{\TraceStepOne}
  \end{columns}
\end{frame}

\begin{frame}{Trace ($\ominus$): Example: Step 2 Shift indices}
  \begin{columns}[T,onlytextwidth]
    \column{0.48\textwidth}
  \centering
  \resizebox{0.5\linewidth}{!}{\TraceStepOne}

  \column{0.04\textwidth}
  \centering \Large $\Longrightarrow$

  \column{0.48\textwidth}
  \centering
  \resizebox{0.5\linewidth}{!}{\TraceStepTwo}
  \end{columns}
\end{frame}

\begin{frame}{Trace ($\ominus$): Example: Step 3 Join directly}
  \begin{columns}[T,onlytextwidth]
    \column{0.48\textwidth}
  \centering
  \resizebox{0.5\linewidth}{!}{\TraceStepTwo}

  \column{0.04\textwidth}
  \centering \Large $\Longrightarrow$

  \column{0.48\textwidth}
  \centering
  \resizebox{0.3\linewidth}{!}{\TraceStepThree}
  \end{columns}
\end{frame}

\begin{frame}{Trace ($\ominus$): Definition (part 2)}
  Let $f : [X + A \rightarrowtail X + B]$ \\
  Then $f \ominus X : [ A \rightarrowtail B ]$,
  Such that the following properties are satisfied:
  \begin{itemize}
  \item Vanishing
  \\~\\~\\~\\~\\~\\
  \item Superimposing
  \\~\\~\\~\\~\\~\\
  \end{itemize} $\qquad (\forall A,B,X : \mathbb{N})$
\end{frame}

\begin{frame}{Trace ($\ominus$): Definition (part 3)}
  Let $f : [X + A \rightarrowtail X + B]$ \\
  Then $f \ominus X : [ A \rightarrowtail B ]$,
  Such that the following properties are satisfied:
  \begin{itemize}
  \item Tightening
  \\~\\~\\~\\~\\~\\
  \item Sliding
  \\~\\~\\~\\~\\~\\
  \end{itemize} $\qquad (\forall A,B,X : \mathbb{N})$
\end{frame}

\begin{frame}{Results}
  \large
  \begin{itemize}
  \item Constructed the category of injective functions \textrm{Inj}.
  \item Defined a tensor product, showing that it preserves identity
    and composition
  \item Defined a trace operation
  \item Proven 'vanishing' on the trace.
  \item Partial construciton of monoidal category
  \item Partial proof of superimposing axiom
  \end{itemize}
\end{frame}

\begin{frame}{Future Work}
  \Large
  \begin{itemize}
    \item Finish completion of monoidal construction
    \item Express the remaining trace axioms formally
    \item Complete the \textrm{Int}-construction
    \item 'Upstream' results to Agda Cubical library.
  \end{itemize}
\end{frame}

\begin{frame}{Thank you}
  \Huge
  Thank you for listening
\end{frame}
\end{document}
