\chapter{Related Work} \label{lit-review}

This chapters surveys related concepts to the notion of virtual sets
discussed in this dissertation.

\subsection{Traced Monoidal Categories in the Literature}
\subsubsection*{Abstract}
This abridged literature review examines ten significant articles on traced
monoidal categories, summarising each and classifying them into four classes:
foundations, extensions, theorems, and applications.
\subsubsection*{Background}
Introduced in 1996 by Joyal, Street, and Verity (JSV)
\cite{joyal1996-traced-monoidal-categories}, traced monoidal categories are a
categorical abstraction of feedback, recursion, and cyclic phenomena. They
generalise ideas from quantum algebra and knot theory to other domains
\cite{reshetikhin1990-ribbon-graphs-invaraints}. A traced monoidal category is a
monoidal category with a trace operator (defined below)
\cite{hu2021-traced-monoidal-categories}. The $\mathrm{Int}$-construction
enables the embedding of traced categories inside compact closed ones
\cite{joyal1996-traced-monoidal-categories}. Subsequent algebraic work extends
traced categories to profunctor bicategories
\cite{hu2021-traced-monoidal-categories}, Hopf monads
\cite{bruguieres2011-hopf-monads-monoidal}, $*$-autonomous categories
\cite{hajgato2013-traced-autonomous-categories}, and pivotal categories
\cite{etingof2005-fusion-categories}. Figure~\ref{fig:litmap} shows a citation graph of the
10 papers discussed, arranged on a plot of citation count against publication
date.
\begin{figure}[ht]
	\centering
	\includegraphics[width=\textwidth]{comp4037-cw3-litmap.png}
	\caption{Citation map of the discussed papers plotted by citation count vs.\ publication date.\protect\footnote{Chart created using \texttt{litmaps.com}.}}
	\label{fig:litmap}
\end{figure}
\subsubsection*{Definitions}
\paragraph{Traced Monoidal Category.}
A \emph{right-traced monoidal category} is a monoidal category $\mathcal{C}$ equipped with natural transformations
$$
	\mathrm{tr}_{R}^{X} : \mathcal{C}[A \otimes X,\, B \otimes X] \Rightarrow \mathcal{C}[A,B]
	\qquad \forall A,B,X : \mathcal{C},
$$
subject to the four axioms in \cite{joyal1996-traced-monoidal-categories}.
\paragraph{Variants and Related Categories.}
\begin{itemize}
	\item \textbf{Left/right traced:} trace ranges over the left or right
	      tensor factor; having both yields a \emph{traced} category
	      \cite{joyal1996-traced-monoidal-categories}.
	\item \textbf{Spherical category:} a traced category in which left and
	      right traces agree \cite{barrett1999-spherical-categories}.
	\item \textbf{Braided category:} traced category with a braiding $c_{X,Y} : X \otimes Y \to Y \otimes X$ \cite{joyal1993-braided-tensor-categories}.
	      % \item \textbf{Symmetric monoidal category:} braided traced category where
	      %       $c_{Y,X}\circ c_{X,Y}=\mathrm{id}_{X\otimes Y}$
	      %       \cite{sharma2018-symmetric-monoidal-categories}.
	\item \textbf{Balanced monoidal category:} braided category plus a twist
	      $\theta_X : X \to X$, subject to extra conditions
	      \cite{joyal1991-geometry-tensor-calculus-2}.
	\item \textbf{Tortile (ribbon) category:} special case of balanced monoidal
	      category in which every object has a right dual
	      \cite{shum1994-tortile-tensor-categories}.
	\item \textbf{Compact closed category:} a symmetric monoidal category where
	      every object has a dual \cite{day1977-note-compact-closed}.
	\item \textbf{$*$-Autonomous category:} symmetric monoidal closed with a
	      dualising object $\bot$ \cite{barr2006-autonomous-categories}; traced
	      $*$-autonomous categories are compact closed
	      \cite{hajgato2013-traced-autonomous-categories}.
	\item \textbf{Hopf monad:} categorical generalisation of Hopf algebras
	      \cite{bruguieres2011-hopf-monads-monoidal,bruguieres2007-hopf-monads,hasegawa2023-traced-monads-hopf}.
\end{itemize}
\subsubsection*{Papers}
\paragraph{Reshetikhin \& Turaev (1990): \emph{Ribbon graphs and their invariants derived from quantum groups} \cite{reshetikhin1990-ribbon-graphs-invaraints}.}
Introduces ribbon categories and their graphical calculus, laying groundwork for twist and braiding in monoidal categories; constructs invariants for ribbon graphs using quantum groups. This underpins graphical representations relevant to traced structures; see also \cite{matsuo2024-quantum-toroidal-algebras}.
\paragraph{Joyal \& Street (1993): \emph{Braided tensor categories} \cite{joyal1993-braided-tensor-categories}.}
Develops theory of braided and tortile (ribbon) monoidal categories, providing coherence and graphical calculus, and introducing structure (e.g.\ compact braided monoidal groupoids) vital for contexts where traces interact with braiding.
\paragraph{Joyal, Street \& Verity (1996): \emph{Traced monoidal categories} \cite{joyal1996-traced-monoidal-categories}.}
Seminal axiomatization of traced monoidal categories, abstracting feedback in symmetric monoidal settings. Main contribution is the $\mathrm{Int}$-construction embedding traced categories in compact closed ones, together with trace axioms. Entirely theoretical with diagrammatic proofs; foundational for later developments.
\paragraph{Simpson \& Plotkin (2000): \emph{Complete axioms for categorical fixed-point operators} \cite{simpson2000-complete-axioms-categorical}.}
Axiomatise fixed-point (iteration) operators for recursion/feedback in categorical semantics; prove completeness results connecting domain-theoretic models and categorical iteration, informing traced reasoning about recursion (cf.\ \cite{bloom1989-equational-logic-iterative}).
\paragraph{Hasegawa (2004): \emph{The algebraic and uniformity principle for traced monoidal categories} \cite{hasegawa2003-uniformity-principle-traced}.}
Generalises Plotkin’s uniformity principle to traced monoidal categories, providing a method to transport and construct traced structure, influencing diagrammatic reasoning \cite{bonchi2025-diagrammatic-algebra-program,ponto2014-traces-symmetric-monoidal}.
\paragraph{Hasegawa (2009): \emph{On traced monoidal closed categories} \cite{hasegawa2009-traced-monoidal-closed}.}
Extends the JSV structure theorem: every traced monoidal category embeds as a full subcategory of a tortile monoidal category via $\mathrm{Int}$. Shows traced monoidal categories are closed iff the canonical inclusion into $\mathrm{Int}$ has a right adjoint.
\begin{figure}[ht]
	\centering
	\includegraphics[width=\textwidth]{comp4037-cw3-hasegawa2009-int.png}
	\caption{The $\mathrm{Int}$-construction on two arrows $f$ and $g$ (adapted from \cite[Sec.\ 3.3]{hasegawa2009-traced-monoidal-closed}).}
	\label{fig:hasegawa-int}
\end{figure}
\paragraph{Brugui`eres, Lack \& Virelizier (2011): \emph{Hopf monads on monoidal categories} \cite{bruguieres2011-hopf-monads-monoidal}.}
Generalises Hopf monads to arbitrary monoidal categories, unifying Hopf-algebraic structures via bimonads with invertible fusion (see also \cite{day2004-quantum-categories-star,bohm2018-hopf-algebras-generalizations}). Foundational for later connections to traced monads and lifted traces \cite{shimizu2019-non-degeneracy-conditions-braided}.
\paragraph{Hajgat\'o \& Hasegawa (2013): \emph{Traced $*$-autonomous categories are compact closed} \cite{hajgato2013-traced-autonomous-categories}.}
Shows every traced $*$-autonomous category is compact closed. Via a key proposition (their Prop.\ 1), concludes the essential equivalence between: traced $*$-autonomous categories, compact closed categories with a dualising object, and $*$-autonomous categories with invertible linear distributivity.
\paragraph{Hu \& Vicary (2021): \emph{Traced monoidal categories as algebraic structures in Prof} \cite{hu2021-traced-monoidal-categories}.}
Characterises traced monoidal categories as traced pseudomonoids in the bicategory $\mathbf{Prof}$, providing an algebraic perspective and equivalence theorems supported by a graphical calculus; clarifies relationships to $*$-autonomous structure.
\paragraph{Hasegawa \& Lemay (2023): \emph{Trace-coherent Hopf monads} \cite{hasegawa2023-traced-monads-hopf}.}
Introduces trace-coherent Hopf monads and gives conditions for lifting a trace to Eilenberg--Moore categories \cite{eilenberg1965-adjoint-functors-triples}, extending traced structure in new algebraic contexts.
\subsubsection*{Categorisation}
\begin{itemize}
	\item \textbf{Foundations}: Introduce axioms and basic properties of traced monoidal categories.
	\item \textbf{Extensions}: Generalise traced monoidal categories to broader algebraic structures.
	\item \textbf{Theorems}: Prove conjectured claims or structural equivalences.
	\item \textbf{Applications}: Use traced monoidal categories in other domains.
\end{itemize}
\begin{table}[h]
	\centering
	\begin{tabular}{|l|c|l|l|l|}
		\hline
		\textbf{Paper (First Author)} & \textbf{Year} & \textbf{Title}                                        & \textbf{Category} & \textbf{Citation}                               \\
		\hline
		Reshetikhin et al.            & 1990          & Ribbon Graphs and their Invariants\ldots              & Foundations       & \cite{reshetikhin1990-ribbon-graphs-invaraints} \\
		Joyal et al.                  & 1993          & Braided Tensor Categories                             & Foundations       & \cite{joyal1993-braided-tensor-categories}      \\
		Joyal et al.                  & 1996          & Traced Monoidal Categories                            & Foundations       & \cite{joyal1996-traced-monoidal-categories}     \\
		Simpson et al.                & 2000          & Complete Axioms for Categorical Fixed-Point Operators & Applications      & \cite{simpson2000-complete-axioms-categorical}  \\
		Hasegawa                      & 2003          & Uniformity Principle for TMC                          & Theorems          & \cite{hasegawa2003-uniformity-principle-traced} \\
		Hasegawa                      & 2009          & On Traced Monoidal Closed Categories                  & Extensions        & \cite{hasegawa2009-traced-monoidal-closed}      \\
		Brugui`eres et al.            & 2011          & Hopf Monads on Monoidal Categories                    & Extensions        & \cite{bruguieres2011-hopf-monads-monoidal}      \\
		Hajgat\'o et al.              & 2013          & Traced $*$-Autonomous Categories are Compact Closed   & Theorems          & \cite{hajgato2013-traced-autonomous-categories} \\
		Hu et al.                     & 2021          & TMC as Algebraic Structures in Prof                   & Extensions        & \cite{hu2021-traced-monoidal-categories}        \\
		Hasegawa et al.               & 2023          & Trace-Coherent Hopf Monads                            & Extensions        & \cite{hasegawa2023-traced-monads-hopf}          \\
		\hline
	\end{tabular}
	\caption{Categorisation of key papers on traced monoidal categories.}
\end{table}
\subsubsection*{Conclusion}
The theory of traced monoidal categories has evolved from basic axiomatisations
to a structure with deep connections to other categories and applications. Core
results such as the $\mathrm{Int}$-construction and ties to compact closed and
ribbon categories have made this an active research area. Continued work is
likely to deepen connections with logic, computation, and quantum theory.
Formalising these structures in Agda (see
\cite{hu2021-formalizing-category-theory}) could further solidify
foundations and provide computational verification of theorems
surveyed here.

\section{Formalization of Categories in Agda}

Hu & Carette's 2021 paper \cite{hu2021-formalizing-category-theory}
first lays out a practical design for implementing and reasoning about
categories in Agda. They discuss the considerations of making a
category formalization that is both extensible and easy to use. Their
approach in MLTT \footnote{MLTT = Martin L{\"o}f Type Theory}
\cite{martin-lof1984-mltt} interestingly they choose to ignore the
higher $\inf$-groupoid structure
\cite{maltsiniotis2010-grothendieck-inf-groupoids} of the hom-sets.

The cubical approach \cite{vezzosi2021-cubical-agda-dependently} is
similar, but requires that the categories are well behaved by being
'set' like. The reason for this is something that I don't have a full
understanding of yet.

There are two major cubical libraries: the standard cubical library
\cite{cubical-agda-lib}, and 1Lab \cite{1lab}. Neither libraries
contain traced monoidal categories so I ported the definition over
from the Agda categories library\cite{agda-categories,hu2021-formalizing-category-theory} to the setting of the standard cubical library.

In my reasearch the formalization of traced monoidal categories in the
Agda categories library is the only one I found in any library or
language, including LEAN mathlib\cite{mathlib2020},
Agda\cite{norell2009-dependently-typed-programming},
Coq\cite{team2021-coq-proof-assistant}, and
Isabel/HOL\nipkow2002-isabelle.
