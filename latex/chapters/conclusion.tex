\chapter{Conclusion}\label{conclusion}

\section{Results}
This dissertation formalizes two category constructions of the
category of finite sets and injective funcitons, providing two
foundations for further development.

\begin{itemize}
  \item We defined two encodings of injective functions:
    \texttt{InjFun}, defined using a dependent sum; and \texttt{Inj},
    defined inductively.
  \item We proved they both form a category
  \item We defined a tensor product for both of these.
  \item We sketched a construction of the monoid laws.
  \item We define a trace operator.
  \item We prove some basic properties (omitted from this report due to time).
\end{itemize}

All definitions were formalized in Cubical Agda, and most of the
lemmas proven were formalized, though there are still significant
holes which require further work to complete.

\section{Discussion}
The central research question (objective \ref{obj:trace-coherence},
and \ref{obj:int-construction}: whether the category \textbf{Inj}
admits a trace satisfying the JSV axioms, which form the foundations
for a full treatment of the \textbf{Int}-construction for compact
closed completionremains only partially answered here. In practice,
the proof burden fell on a long tail of elementary lemmas about finite
sets, sums, and path manipulations. Midway through, I pivoted from the
dependent-sum encoding to an inductive encoding in the hope of
simplifying coherence proofs; however, both approaches incurred
distinct overheads (transport management vs.\ structural bureaucracy),
and neither reached a full monoidal (let alone traced) package within
the project time-frame.

This project turned out to be much more challenging than initially
anticipated. Neither of my approaches gave a simple way to go about
the formalization of the result that injective maps on finite sets
form a traced monoidal category. Progress was limited by the large
amount of lemmas and sub-lemmas that need to be completed. After
feeling like the progress had slowed with the depdent sum half-way
though the project, I thought that starting again with an inductive
approach would be significantly simpler. It turns out this assessment
was wrong and both approaches had significant limitations that I was
unable to overcome to complete the construction of even a symmetric
monidal category let alone a traced monoidal category.

If I was doing this project again, I would shrink the scope
significantly and commit to a single representation for the duration
of the project. This would have been better for either approach if I'd
stuck with it and proven the monoidal axioms before attempting traced
monoidal categories and implementing the \textrm{Int}-construction
\cite{joyal1996-traced-monoidal-categories}.

Additionally I noticed that I hadn't allocated enough time to do the
report justice, and I started it too late. I wasn't able to finish
writing up the formalization of the tensor product on `Inj`, because I
spent too much time trying to make progress on the code, to have
something impressive to show at the end.

The project was very engaging, and I would be very interested in
trying to complete the construction after I have submitted this
dissertation.
