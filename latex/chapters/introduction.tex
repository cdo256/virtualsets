% =============================
% Chapter 1: Introduction
% =============================
\chapter{Introduction}
\section{Motivation and Scope}

The motivation for this project is mainly exploritory. It was observed
by Mark Williams (CITE preprint) that the category of finite sets and
injective function (\textrm{Inj}) admits a trace operator, as
introduced in Joyal-Steet-Verity et. al. (CITE). A trace operator acts
like a subtraction on morphisms, and in the case of finite sets, it
literally removes elements from the map, decreasing the size of the
domain and codomain by the same amount.

The intuition is then that if we have a tensor that acts as an
addition, and a trace acting as a subtract operator, then can we
construct an extension that is closed under traced using these
operations? To put another way, can we construct a category in
which every map has an inverse. It was proven by JSV CITE that a
category with a trace satisfying certain coherence conditions XREF can
be used to construct a category closed under trace in which the
original category forms a sub-category. In other words, it forms an
extension to the original category, that is closed under trace.

This construction has not been formalized to date in any major proof
assistant (Agda, Coq, Isabel/HOL, LEAN) CITE. The original plan was to
implement it here, however it was found after some time developing
this, that the process was more complex that initially expected.

\section{Informal Problem Statement}
At a high level, this dissertation investigates whether an injective function
on finite sets can be operated on by a certain 'trace' operator, satisfies the
Joyal-Stree-Verity axioms for a traced monoidal category, and hence a compact
closed category, using the Int-construction.

\section{Structure of the Dissertation}
% Introduction
% Preliminaries
% Literature Review
% Methodology
% Formalization
% Conclusion
