\chapter{Introduction} \label{introduction}
\section{Motivation and Scope}
The motivation for this project is mainly exploritory. It was observed
by a preprint by Mark Williams (2025) \cite{williams2025-vsets} that the
category of finite sets and injective function (\textrm{Inj}) admits a
trace operator, as introduced in Joyal-Steet-Verity
(1996)\cite{joyal1996-traced-monoidal-categories} . A trace operator
acts like a subtraction on morphisms, and in the case of finite sets,
it literally removes elements from the map, decreasing the size of the
domain and codomain by the same amount.

The intuition is then that if we have a tensor that acts as an
addition, and a trace acting as a subtract operator, then can we
construct an extension that is closed under traced using these
operations? To put another way, can we construct a category in
which every map has an inverse. It was proven by Joyal
et. al. (1996) \cite{joyal1996-traced-monoidal-categories} that a
category with a trace satisfying certain coherence conditions in
section \ref{traced-monoidal-categories} can
be used to construct a category closed under trace, called a compact
closed category, in which the original category forms a
sub-category. In other words, it forms an extension to the original
category, that is closed under trace.

Although traced monoidal categories have been expressed in the Agda
categories library\cite{agda-categories} the \textrm{Int}-construction (as
defined in Joyal et. al. 1996 \cite{joyal1996-traced-monoidal-categories})
has not been formalized to date in any major proof assistant
(Agda\cite{vezzosi2021-cubical-agda-dependently},
Coq\cite{team2021-coq-proof-assistant},
Isabell/HOL\cite{nipkow2002-isabelle}, LEAN
Mathlib\cite{mathlib2020}). The original plan was to implement it
here, however it was found after some time developing this, that the
process was more complex that initially expected.

\section{Informal Problem Statement}
At a high level, this dissertation investigates whether an injective
function on finite sets can be operated on by a certain 'trace'
operator, satisfies the Joyal-Street-Verity axioms
\cite{joyal1996-traced-monoidal-categories} for a traced monoidal
category, and hence a compact closed category, using the
Int-construction.

\section{Research Objectives}

The aim of this project is to build a up a definition of the category of 'virtual
sets' by constructing a category of injective functions, defining a tensor and
trace opoerator, and proving that the coherence conditions hold, and finally to
use the \textrm{Int}-construction to build a compact closed category, which is a
symmetric monoidal category in which every object has a dual.

We list seven objectives, each of which building of the previous objective: 

\begin{objective}[Define the 'precategory' \textrm{Inj}] \label{obj:precategory}
  Build the precategory \cite{maclane1998-categories-book} \textrm{Inj}:
  \begin{itemize}
  \item[\textbf{objects}] Define the type finite sets
  \item[\textbf{morphism}] Formally define injective functions
  \item[\textbf{identity}] Define an identity $\mathit{id}\ :\ A \to A$ morphism for every object.
  \item[\textbf{composition}] Define a composition operator $\circ$ on morphisms:
    $$
    g \circ f \text{ is an arrow from $A$ to $C$ where } f : A \to B \text{ and } g : B \to C
    $$
  \end{itemize}
\end{objective}
\begin{objective}[Show that \textrm{Inj} forms a (full) category] \label{obj:category}
  Show that the precategory defined in objective \ref{obj:precategory} forms a
  (full) category \cite{awodey2010-category-theory-book}. This requires the following:
  \begin{itemize}
  \item[\textbf{associativity}] show that for suitable morphism types $f,g,h$ that both ways of composing results in the same morphism:
    $$
    h \circ (g \circ f) = (h \circ g) \circ f
    $$
    \begin{align*}
      \text{where } & f : A \to B
      & g : B \to C
      & h : C \to D
      & A, B, C, D \text{ objects}
    \end{align*}
  \item[\textbf{unit}]
    \begin{align*}
    \mathit{id} \circ f = f \text{ and,}\\ 
    f \circ \mathit{id} = f
    \end{align*}
    \begin{align*}
      \text{where } & f : A \to B
    \end{align*}
  \end{itemize}

  Build the category \textrm{Inj}, with a suitable composition function $\circ$ and
  identity element $\mathrm{id}$. Construct a proof of associativity of and
  left and right unitor laws (cf. common texts include MacLane 1998 and Awodey
  2010 \cite{maclane1998-categories-book,awodey2010-category-theory-book}).
  Finally show that the type \textit{Inj} is set-like in the homotopy sense (see definition \ref{def:set}).
\end{objective}
\begin{objective}[Define a tensor operator] \label{obj:tensor}
  Define a tensor operator $\oplus$ and unit $𝟘$ on \textrm{Inj}.
  That is a *bifunctor* $F$ from $\textrm{Inj}^2$ to \textrm{Inj} (see definition \ref{def:functor}) and an object selected to act as the unit.
  It must be checked that the functor $F$ preserve the following operations:
  \begin{itemize}
  \item[\textbf{dom/cod}] Given a morphisms $f$, then, \\
    the domain $\mathrm{dom}(F f) = F \mathrm{dom f}$, and \\
    the codomain $\mathrm{cod}(F f) = F \mathrm{cod f}$. \\
    (NB. This is handled by the type system in Agda.)
  \item[\textbf{identity}] A functor must map identity arrows to identity arrows.
  \item[\textbf{composition}] $F$ applied to the composition $g \circ f$ is equal to $F g \circ F f$ for arrows $f,g$.
  \end{itemize}
\end{objective}
\begin{objective}[Construct a (weak) Monoidal Category] \label{obj:monoidal}
  Show that the tensor product defined above forms a monoidal
  category. This required the following natural isomorphisms:
  \begin{itemize}
  \item[\textbf{associator}] $\alpha_{A,B,C} : A \oplus (B \oplus C) \cong (A \oplus B) \oplus C$.
  \item[\textbf{left-unitor}] $\eta_{A} : 𝟘 \oplus A = A$
  \item[\textbf{right-unitor}] $\rho_{A} : A \oplus 𝟘 = A$
  \end{itemize}
  Then show that the following two coherence laws hold: \\  
  \begin{itemize}
  \item[\textbf{'triangle' law}] when there are two ways to use
    unitors to to cancel the unit, then the two resulting morphisms
    are equal.
  \item[\textbf{'pentagon' law}] When there are two different ways to
    convert between $(A \oplus (B \oplus C)) \oplus$ and
    $(A \oplus B) \oplus (C \oplus D)$, using the associator $\alpha$,
    then these arrows must be equivalent.
  \end{itemize}
Aside: It turns out that these two laws are sufficient to say that if there
is any way to convert from one object to another using just the
unitors, and associators, then it must be unique. (Mac
Lane 1998 \cite{maclane1998-categories-book} in Ch VII § 2.)
\end{objective}
\begin{objective}[Define a trace operator on \texttt{Inj}]
  Define a function $Tr_X : ((A \oplus X) ↣ (B \oplus X)) \to (A ↣ B)$. See definition \ref{def:trace}.
\end{objective}
\begin{objective}[Check trace coherence laws]
  \label{obj:trace-coherence}
  The following laws must hold:
  \begin{itemize}
    \item[tightening] See axiom \ref{axiom:tightening}.
    \item[sliding] See axiom \ref{axiom:sliding}.
    \item[vanishing] See axiom \ref{axiom:vanishing}.
    \item[superimposing] See axiom \ref{axiom:superimposing}.
  \end{itemize}
\end{objective}
\begin{objective}[Implement \textrm{Int}-construction]
  \label{obj:int-construction}
  Implement the construcion of a compact closed
  category\cite{day1977-note-compact-closed} from a traced monoidal
  category. This involves reproducing the results from Joyal
  et. al. 1996 \cite{joyal1996-traced-monoidal-categories}.
\end{objective}

\begin{figure}[h]
  \centering
    \begin{tikzcd}
      (A \oplus 𝟘) \oplus B \arrow[rr, "\alpha_{A,𝟘,B}"] \arrow[dr, "\lambda_A \oplus \mathit{id}_B"'] & &
      A \oplus (𝟘 \oplus B) \arrow[dl, "\mathit{id}_A \oplus \rho_B"] \\
      & A \oplus B &
    \end{tikzcd}
  \caption{The triangle coherence law for a monoidal category (\(\mathcal{C}, \oplus, 𝟘\)).} \cite{awodey2010-category-theory-book,egbert1998-coherence-theorem}
  \label{fig:monoidal-triangle}
\end{figure}

\begin{figure}[h]
  \centering
    \begin{tikzcd}[column sep=small]
      & ((A \oplus B) \oplus C) \oplus D \arrow[dr, "\alpha_{A\oplus B, C, D}"] \\
      (A \oplus (B \oplus C)) \oplus D \arrow[ur, "\alpha_{A, B, C} \oplus \mathit{id}_D"] \arrow[d, "\alpha_{A, B\oplus C, D}"'] 
      && (A \oplus B) \oplus (C \oplus D) \\
      A \oplus ((B \oplus C) \oplus D) \arrow[rr, "\mathit{id}_A \oplus \alpha_{B, C, D}"'] 
      && A \oplus (B \oplus (C \oplus D)) \arrow[u, "\alpha_{A, B, C\oplus D}"']
    \end{tikzcd}
  \caption{The pentagon coherence law for a monoidal category (\(\mathcal{C}, \oplus, 𝟘\)). \cite{awodey2010-category-theory-book,egbert1998-coherence-theorem}}
  \label{fig:monoidal-pentagon}
\end{figure}

\section{Structure of the Dissertation}

We begin this dissertation by defining some category-theoretic
definitions we will be relying on in chapter \ref{prelim}. In chapter \ref{lit-review},
we perform a brief literature review of traced monoidal categories. In
chapter \ref{formalization}, we build up the formalization of
injective functions up to defining a trace operator. 
Following on from there, in  we construct the formalization of the
structures in two streams, each starting from the defintion and
properties on finite sets, building up to category theoretic
definitions. The first stream builds injective functions as a
dependent sum of a function and a proof that function is injective,
and the second stream defines functions inductively, which are
injective by construction. Finally, in chapter \ref{conclusion} we
list the results, discuss further work, and limitations of this
formalization effort.

\section{Methodology}

The majority of this work was carried out in a single Git
\cite{chacon2014-git} repository hosted on 
\href{https://github.com/cdo256/virtualsets}{GitHub}
\cite{odonnell2025-virtual-sets}. Is is implemented in cubical Agda
\cite{vezzosi2021-cubical-agda-dependently} using the cubical Agda
standard library \cite{cubical-agda-lib}. Originally I started working
in the Agda standard library \cite{agda-stdlib} however it was
suggested by Thorsten that I should switch to a cubical library, to
avoid "setoid hell" (cf. Altenkirch 2017
\cite{altenkirch2017-setoid-hell}, more generally, Allais et. al 2025
\cite{allais2025-frex}), suggesting 1Lab library \cite{1lab}. I
moved my definitions over to 1Lab but there were some akward
definitions, which were slowing me down. Most notably the definition
of finite sets (\texttt{Fin}), which used a quotient (truncation) type
\cite{huber2017-cubical-canonicity,doorn2016-hit-truncation}. I eventially
settled on the standard cubical library \cite{cubical-agda-lib}.

The work is fully reproducible for its Git repository
\cite{odonnell2025-virtual-sets}, using Nix flakes \cite{nix-flakes},
and include all of depdencies all of the lemmas I have completed. It
even include the list of dependencies and python script for building
the report. The code is structured modularly and designed
intentionally for ease of further development.

The layout of the repository is as follows:

\dirtree{%
.1 build.py.
.1 flake.lock.
.1 flake.nix.
.1 justfile.
.1 latex.
.2     agda.sty.
.2     chapters.
.3         conclusion.tex.
.3         formalization.tex.
.3         introduction.tex.
.3         lit-reivew.tex.
.3         prelim.tex.
.2     main.tex.
.2     preamble.tex.
.2     report.bib.
.1 libraries.
.1 nix.
.2     args.nix.
.2     packages.nix.
.2     shells.nix.
.1 postprocess-latex.pl.
.1 src.
.2     Compat.
.3         1Lab.
.3         1Lab.agda.
.2     Dissertation.
.2     DissertationTex.
.2     Notes.
.2     VSet.
.3         All.agda.
.3         Cat.
.4             Inj.agda.
.4             InjFun.agda.
.4             Trace.agda.
.3         Data.
.4             Fin.
.4             Fin.agda.
.4             HITTree.
.4             Inj.
.4             InjFun.
.4             Maybe.agda.
.4             Nat.
.4             Nat.agda.
.4             NatPlus.
.4             Sum.
.4             SumTree.
.4             Tree.
.4             VecFun.
.3         Function.
.4             Base.agda.
.4             Injection.agda.
.4             Iso.agda.
.4             Properties.agda.
.3         Path.agda.
.3         Prelude.agda.
.3         Relation.
.4             Definitions.agda.
.4             WellFounded.
.3         Transform.
.4             Inj.
.5                 Compose.
.5                 Elementary.
.5                 Inverse.
.5                 Tensor.
.5                 Trace.
.4             InjFun.
.5                 Compose.agda.
.5                 Flattern.agda.
.5                 Inflate.agda.
.5                 Pred.agda.
.5                 Properties.agda.
.5                 Sub.agda.
.5                 Tensor.agda.
.5                 Unused.agda.
.1 virtual-set.agda-lib.
}

There are 3 main top-level directories for different
purposes:

\texttt{./latex/} defines the parts of this dissertation
that are written in \textit{TeX} or an ajacent language, such as
\textit{BibTeX}. These are assembled by `./build.py` with the sections
of the dissertation written in literate Agda (with markdown), to
produce the final PDF file. Everything in \texttt{./src/Dissertation/}
is converted into TeX with Pandoc, in order to process the embedded
TeX commands, then Agda type checks the file, before adding syntax
hylighting and references before inserting it into
\texttt{./latex/generated/}, where it in turn is referred to by
\texttt{./latex/chapters/formalization.tex}. This circuitous route was
required as to date, Agda does not currently have a markdown-to-latex
backed.

\texttt{./nix/} together with \texttt{./flake.nix} and
\texttt{./flake.lock}.

Finally all of the Agda source is contained in \texttt{./src}. Compat
contains functions that allow the use the functions of other
libraries, such as 1Lab. These are not source I have developed myself,
but they are functions copied into the repo to enable use of certain
functions from other libraries. Normally libraries can't be mixed,
because it can result in mismatch of primitive definitions, meaning
you have to convert between these definitions manually every time you
want to use a function from a foreign
library. \texttt{./src/Dissertation/} and
\texttt{./src/DissertationTex/} contins the files to generate the
formalization chapter.

The bulk of the source is in \texttt{./src/VSet/}:

\begin{itemize}
  \item[\texttt{./src/VSet/All.agda}] is a list of all files to type
    check to ensure the repository is still valid. I run this
    regularly to ensure I haven't broken anything.
  \item[\texttt{./src/VSet/Cat/}]: This contains category definitions
    and category constructions. Inj.agda and InjFun.agda are the two
    category constructions for the two ways to define injective maps
    explored in this work.
  \item[\texttt{./src/VSet/Data/}]: This directory contains various
    definitions and properties of Data types. Most properties of data
    types are defined here, except the main development for Inj and
    InjFun is performed inside \texttt{./src/VSet/Transform}.
  \item[\texttt{./src/VSet/Function.agda}, and \texttt{./src/VSet/Path.agda}] define basic properties of functions and paths, for use elsewhere.
  \item[\texttt{./src/VSet/Relation/}] defines basic properties of functions and paths, for use elsewhere.
  \item[\texttt{./src/VSet/Transform/}] defiens transformations of
    injective functions, such as inserting/removing links up to
    describing a trace function. This is where the majority of the
    development is situated.
\end{itemize}
  
