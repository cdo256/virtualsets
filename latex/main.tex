\documentclass[4pt,a4paper]{report}
\usepackage{amsmath,amssymb}
\usepackage{graphicx}
\usepackage{hyperref}
\usepackage{amsthm}
\usepackage{amsfonts}
\usepackage{cite}
\usepackage{tikz-cd}
\usepackage{stmaryrd}
\usepackage{listings}
\usepackage{xcolor}
\usepackage{fontspec}
\usepackage[references]{latex/agda}
\usepackage{unicode-math}
\usepackage{tikz}
\usetikzlibrary{positioning,chains,fit,shapes,calc,arrows.meta}
\usepackage{ucharclasses}
\usepackage{newunicodechar}

% Font setup
\setmainfont{DejaVu Serif}
\setsansfont{Liberation Sans}
\setmonofont{JuliaMono}
\let\oldttfamily\ttfamily
\renewcommand{\ttfamily}{\footnotesize\oldttfamily}

% Agda code style
\renewcommand{\AgdaCodeStyle}{%
  \setmainfont{JuliaMono}
  \setsansfont{JuliaMono}
  \setmonofont{JuliaMono}
  \footnotesize % Make the code smaller to fit within width.
}
\setlength{\mathindent}{0pt}

% TikZ helper
\newcommand{\dotrow}[3]{% end, label, labelpos
  \begin{scope}[start chain=going right]
    \foreach \i in {0,...,#1}
      \node[on chain, dotnode, label=#3:{\i}] (#2\i) {};
  \end{scope}
}

\tikzset{
  dotnode/.style = {circle, fill, inner sep=0.5mm},
  thickedge/.style  = {thick},
  largestep/.style = {
    ultra thick,
    -{>[length=8mm, width=8mm]},
    shorten >=1cm,
    shorten <=1cm},
  pivot/.style={circle, fill=white, inner sep=0.5mm, draw=red}
}

% Theorem styles/environments
\theoremstyle{plain}
\newtheorem{theorem}{Theorem}
\newtheorem{lemma}{Lemma}
\newtheorem{axiom}{Axiom}
\theoremstyle{definition}
\newtheorem{definition}{Definition}
\newtheorem{conjecture}{Conjecture}
\theoremstyle{remark}
\newtheorem{note}{Note}
\newtheorem{question}{Research Question}

% Pandoc expects this to be defined.
\providecommand{\tightlist}{%
  \setlength{\itemsep}{0pt}\setlength{\parskip}{0pt}
}


\title{Toward a Formalization of 'Virtual Sets'}
\author{Christina O'Donnell}
\subtitle{School of Computer Science, University of Nottingham}
\date{September 2025}
\begin{document}

\maketitle

\section*{Acknowledgements}
I would like to thank Thorsten Altenkirch
for helping direct this project and for his invaluable feedback. I
also must thank Mark Williams for sharing his idea with me that
motivated dissertation.

\section*{Delcaration}
Submitted 21st September 2025, in partial fulfilment of the condiations for the award of the degree of MSc (Computer Science).

\begin{abstract}
  This dissertation investigates the categorical structure underlying
  the monoidal category of finite sets and injective functions, with
  the goal of understanding whether it admits a well-defined trace
  operator. A trace is a categorical generalization of feedback
  structures such as recursion, or feedback, introduced by Joyal
  et. al. (1996), traces can be applied to perform the role of
  recursion in categorical semantics of programming langauges, and the
  role of feedback in quantum computing. Mark Williams (2025) recently
  conjectured that the category of finite sets and injective functions
  may have a sufficient trace operator, however the construction has
  yet to be formally verified.

  The notion of 'virtual set', introduced by Williams (2025) is a set
  that can contain 'negative elements', satisfying the condition for a
  compact closed category. This construction amounts to taking a
  quotient of $\mathrm{Inj}^2$, and then proving coherence of the
  result, as spelled out in Joyal et. al. (1996). The difficulty is to
  prove all of the coherence lemmas required to show it has monoidal,
  and trace structure, which is independent of equivalent pairs of
  functions. Such a structure is of interest in simplicial
  type theory, where all inclusion map between simplicial complexes
  must necessarily be injective.

  This dissertation proceeds by developing the category \textrm{Inj}
  of finite sets and injective functions, constructing a composition
  operator and proving associativity and unit laws. A tensor product
  is then defined as well as a partial proof of the monoidal laws.
  We then define a trace operator and prove some of the required
  properties about its behavior. Finally, the \textrm{Int}
  construction is discussed.

  All definitions and proofs are implemented in cubical Agda, making
  this one of the first attempts to formalize the notion of traced
  monoidal categories. We discuss my two main approaches at
  formalizing the monoidal category of injective functions, and
  constructing the tensor operator, and trace, as well as proving some
  of the properties about it. \\
  \textbf{Keywords}: Trace Monoidal Categories, Compact Closed Categories, Cubical Agda 
\end{abstract}
\tableofcontents
\listoffigures

% =============================
% Chapter 1: Introduction
% =============================
\chapter{Introduction}
\section{Motivation and Scope}

\section{Informal Problem Statement}
At a high level, this dissertation investigates whether an injective function
on finite sets can be operated on by a certain 'trace' operator, satisfies the
Joyal-Stree-Verity axioms for a traced monoidal category, and hence a compact
closed category, using the Int-construction.

\section{Structure of the Dissertation}
% Introduction
% Preliminaries
% Literature Review
% Methodology
% Formalization
% Conclusion

\chapter{Preliminaries} \label{prelim}
\section{Category Theory}
\subsection{Introduction}
Category theory is the study of abstract structure generalizing
concepts seen across many disciplines such as mathematics, computer
science and physics. \cite{maclane1998-categories-book,awodey2010-category-theory-book} 
\begin{definition}[Category \cite{hott2013-book, awodey2010-category-theory-book}]
	From the perspective of type theory, \emph{category} consists
	of the following data:
	\begin{itemize}
		\item a type of objects $\mathsf{Obj} : \mathcal{U}$;
		\item for every $A,B:\mathsf{Obj}$, a type of morphisms
		      $\mathsf{Hom}(A,B) : \mathcal{U}$;
		\item for every $A:\mathsf{Obj}$, an identity morphism
		      $\mathsf{id}_A : \mathsf{Hom}(A,A)$;
		\item a composition operation
		      \[
			      \circ \;:\; \prod_{A,B,C:\mathsf{Obj}}
			      \mathsf{Hom}(B,C) \to \mathsf{Hom}(A,B) \to \mathsf{Hom}(A,C).
		      \]
	\end{itemize}
	These are equipped with terms demonstrating that:
	$$
		\begin{aligned}
			 & \mathsf{assoc}_{h,g,f} \;:\;
			(h \circ g) \circ f \;=\; h \circ (g \circ f),                  \
			 & \mathsf{idl}_{f} \;:\; \mathsf{id}_B \circ f \;=\; f, \qquad
			\mathsf{idr}_{f} \;:\; f \circ \mathsf{id}_A \;=\; f,
		\end{aligned}
	$$
	for all $A,B,C,D:\mathsf{Obj}$ and
	$f:\mathsf{Hom}(A,B)$, $g:\mathsf{Hom}(B,C)$, $h:\mathsf{Hom}(C,D)$.
	Here $=$ denotes the identity type of the ambient type theory.
\end{definition}
Classically objects are considered to be collection of objects with some structure, and morphisms considered structure preserving maps. However there is a wide array of interpretations are possible. To list a few:
\begin{table}
	\begin{tabular}{|l|l|l|}
		\hline
		\textbf{Category} & \textbf{Objects} & \textbf{Morphisms} \\
			\hline
			$\mathbf{Set}$                                                                                                                                       & Sets               & Maps between sets                 \\
			\hline
			$\mathbf{Ord}$                                                                                                                                       & Total orders       & Order preserving maps             \\
			\hline
			$\mathbf{Type}$ \footnote{Note that this name is not universal, and there is no consensus as to what definition to use to define this specifically. See for example Univalent Foundations HoTT book §9.9 The Rezk Completion.} & Types              & Computable functions between types\\
			\hline
			$\mathbf{Grp}$                                                                                                                                       & Groups             & Group homomorphisms               \\
			\hline
			$\mathbf{Top}$                                                                                                                                       & Topological spaces & Continuous maps                   \\
			\hline
			$\mathbf{Cat}$                                                                                                                                       & 'small' categories & Functors                          \\
			\hline
			$\mathbf{FinSet}$                                                                                                                                    & Finite Sets        & maps                              \\
			\hline
	\end{tabular}
	\caption{Examples of common categories. Standard definitions and names come from \cite{maclane1998-categories-book,awodey2010-category-theory-book,leinster2014-basic-category-theory}.}
\end{table}
\begin{note}
	Categories are considered 'small' if their objects and morphisms can be
	represented as sets. This is done at the type level, ensuring that the
	category $\mathrm{Cat}$ is not contained in itself, to avoid
	contradictions.
\end{note}
\subsection{Functors}
Functors are mappings between categories that preserve the categorical
structure, allowing us to relate different categorical contexts. A functor $F$
must map identity morphisms to identity morphisms, map morphisms from
A to B to morphisms between $F A$ and $F B$, as well as respecting composition.
\cite{awodey2010-category-theory-book}
\begin{definition}[Functor] \label{def:functor}
	A functor F between categories $\mathcal{C}$ and $\mathcal{D}$ is made up of two components:
	object-part: Function between $obj(C)$ and $obj(D)$
	arrow-part: Function between $arr(C)$ and $arr(D)$
	with the property that compositionality is preserved.
	If $f : x \to y$ and $g : y \to z$ are arrows in C
	Then $F(f) \circ F(g) = F(f \circ g)$ \cite{awodey2010-category-theory-book}
\end{definition}
\begin{definition}[Product Category]
	Given a category C and a category D, the product category, written C x D is constructed of objects made of pairs of $c \in C, d \in D$, and morphisms made from pairs of morphisms $f : c1 \to c2; g : d1 \to d2$ then (f , g) is a morphism from (c1 , d1) to (c2 , d2).
	It can be seen that these satisfy the property of being associative, and having identities.\cite{leinster2014-basic-category-theory}
n\end{definition}
The product operator on categories is not strictly associative in terms of equality, but it is common in category theory to only be interested in isomorphism of objects and in this case isomorphism of categories. We will see later the construction of a monoidal product and
\begin{definition}[Multifunctor] \label{def:multifunctor}
	\begin{itemize}

		\item A \emph{bifunctor} F from $C x C \to D$ is simply a functor from the product category C x C to D.
		\item A \emph{trifunctor} F from $C x C x C \to D$ is likewise a functor from the product category C x C x C to D.
	\end{itemize} \cite{awodey2010-category-theory-book}
\end{definition}
\subsection{Natural Transformations}\cite{maclane1998-categories-book}
Natural transformations provide a way to compare functors and encapsulate the
idea of systematic correspondence between categorical maps.
\begin{definition}[Natural Transformation]
Given a pair of categories C, D and a pair of functor $F, G : C \to D$.
A natural transformation is a family of maps $\eta_{x} : F x \to G x$ for
all $x : C$.
This map must have the following property, called naturality property for
all $x, y : C; f : x \to y$ arrow in C:
$\eta_y \circ F f = G f \circ \eta_x$
\end{definition}

\section{Monoidal Categories}

Monoidal categories are an extension of the notion of a \emph{monoid}, a
category with only one object, but instead of being composed by the category
composition operator $\circ$ and identity arrows $id$, it works on the object
level, having a separate tensor product $\otimes$ and identity object $I$.
These must satisfy the monoid laws, but only up to natural
isomorphisms. For example the associator given by $\alpha_{A,B,C}$,

$$\alpha_{A,B,C} : A \otimes (B \otimes C) \equiv (A \otimes B) \otimes C$$

is an natural isomorphism and doesn't need to be a strict
equality. However it does need to be subject to 'coherence laws' these
are rules that state that certain diagrams commute (which means that
when you have multiple paths to get between certain objects, then, if
the transformations are used then the result will be an identical
morphism. In other words, there's at most one canonical way to convert
between two objects forllowing the transformations.

We now formally state the definition of monoidal category.

\begin{definition}[Monoidal Category \cite{egbert1998-coherence-theorem}]
  Given a category $𝓒$, a monoidal category is made up of the following
  data:

\begin{itemize}
  \item $I : 𝓒$: Identiy object
  \item $\otimes : 𝓒 × 𝓒 → 𝓒$: A product bifunctor.
  \item $\alpha_{A,B,C} : A \otimes (B \otimes C) \equiv (A \otimes B) \otimes C$
  \item $\rho_A : A \otimes I \equiv A$
  \item $λ_A : I \otimes A \equiv A$
  Such that certain coherence conditions hold, spelled out CITE
\end{itemize}

\end{definition}
\section{Traced Monoidal Categories} \label{traced-monoidal-categories}
\begin{definition}[Right Trace]
        \label{def:trace}
	In a traced monoidal category $\mathcal{C}$, a \emph{right trace} is a natural transformation
	$$
		\mathrm{tr}_{R} : \prod_{X : \mathcal{C}} \mathcal{C}[A \otimes X,\, B \otimes X]
		\;\Rightarrow\; \mathcal{C}[A,B]
		\qquad \forall A,B : \mathcal{C}.
	$$
	Equivalently, written with the parameter $X$ as a superscript:
	$$
		\mathrm{tr}_{R}^{X} : \mathcal{C}[A \otimes X,\, B \otimes X]
		\;\Rightarrow\; \mathcal{C}[A,B]
		\qquad \forall A,B,X : \mathcal{C}.
	$$
	Intuitively, the trace can be understood as a feedback loop: part of the output of a morphism is ``fed back'' into ts input.
\end{definition}
A right trace must satisfy the following axioms.
Let $A,B,C,D,X,Y : \mathcal{C}$. (Originally introduced in Joyal et. al. 1996 \cite{joyal1996-traced-monoidal-categories}.) 
\begin{axiom}[Tightening]
        \label{axiom:tightening}
	For morphisms
	$$
		h : A \to B,
		\quad f : B \otimes X \to C \otimes X,
		\quad g : C \to D,
	$$
	we have
	$$
		\mathrm{tr}_{R}^{X}\!\big( (g \otimes 1_X) \circ f \circ (h \otimes 1_X) \big)
		= g \circ \mathrm{tr}_{R}^{X}(f) \circ h.
	$$
\end{axiom}
\begin{axiom}[Sliding]
        \label{axiom:sliding}
	For morphisms
	$$
		f : A \otimes X \to B \otimes Y,
		\quad g : Y \to X,
	$$
	we have
	$$
		\mathrm{tr}_{R}^{X}\!\big( (1_B \otimes g) \circ f \big)
		= \mathrm{tr}_{R}^{Y}\!\big( f \circ (1_A \otimes g) \big).
	$$
\end{axiom}
\begin{axiom}[Vanishing]
        \label{axiom:vanishing}
	\begin{enumerate}
		\item For all $f : A \otimes I \to B \otimes I$,
		      $$
			      \mathrm{tr}_{R}^{I}(f) = f,
		      $$
		      where $I$ is the monoidal unit.
		\item For all $f : A \otimes (X \otimes Y) \to B \otimes (X \otimes Y)$,
		      $$
			      \mathrm{tr}_{R}^{X \otimes Y}(f)
			      = \mathrm{tr}_{R}^{X}\!\big( \mathrm{tr}_{R}^{Y}(f) \big).
		      $$
	\end{enumerate}
\end{axiom}
\begin{axiom}[Superposing]
        \label{axiom:superimposing}
	For morphisms
	$$
		f : A \otimes X \to B \otimes X,
		\quad g : C \to D,
	$$
	we have
	$$
		\mathrm{tr}_{R}^{X}(f) \otimes g
		= \mathrm{tr}_{R}^{X}(f \otimes g)
	$$
\end{axiom}
\begin{definition}[Left trace operator]
A left trace is a trace operator on the opposite functor $⊗⁻$, obtained by swapping the order of the arguments, so $A ⊗⁻ B = B ⊗ A$.
\cite{joyal1996-traced-monoidal-categories}
\end{definition}
\begin{definition}[Planar Category]
A planar category is a category with a left and right trace.
\cite{joyal1996-traced-monoidal-categories}
\end{definition}
\begin{definition}[Symmetric Trace Category]
A planar category is symmetric monoidal. if the cross transformation 
\cite{joyal1996-traced-monoidal-categories}
\end{definition}
\begin{definition}[Spherical Category]
A spherical category is category in which left and right traces agree.
\cite{barrett1999-spherical}
\end{definition}

\begin{definition}[Compact Closed Category]
A compact closed category is a traced monoidal category in which every
object has a dual, which is roughly its inverse with respect to the
monoidal product. Joyal-Stree-Verity introduce the \textbf{Int}-construction.
\cite{joyal1996-traced-monoidal-categories}
\end{definition}

\include{latex/chapters/lit-review}
\chapter{Formalization}

\input{latex/generated/DissertationTex/Intro.tex}

\input{latex/generated/DissertationTex/Basic.tex}
% src/VSet/Prelude.agda
% src/VSet/Path.agda
% src/VSet/Data/Sum/Properties.agda
% src/VSet/Data/Maybe.agda
% src/VSet/Relation/Definitions.agda

% \input{latex/generated/DissertationTex/Nat.tex}
% % src/VSet/Data/Nat/Order.agda
% % src/VSet/Data/Nat/Properties.agda
% 
% % \input{latex/generated/DissertationTex/WellFounded.tex}
% % src/VSet/Relation/WellFounded/Base.agda
% % src/VSet/Relation/WellFounded/Lex.agda
% % src/VSet/Data/Nat/WellFounded.agda
% 
% % \input{latex/generated/DissertationTex/Tree.tex}
% % % src/VSet/Data/Tree/Base.agda
% % % src/VSet/Data/Tree/Properties.agda
% % % src/VSet/Data/SumTree/Base.agda
% % % src/VSet/Data/SumTree/Metrics.agda
% % % src/VSet/Data/SumTree/WellFounded.agda
% % % src/VSet/Data/HITTree/Base.agda

\input{latex/generated/DissertationTex/FinChoice.tex}
 
\input{latex/generated/DissertationTex/Fin.tex}
% src/VSet/Data/Fin/Base.agda
% src/VSet/Data/Fin/Default.agda
%% src/VSet/Data/Fin/Minus.agda
% src/VSet/Data/Fin/Order.agda
 
\input{latex/generated/DissertationTex/InjFun.tex}
% src/VSet/Data/InjFun/Injection.agda
% src/VSet/Data/InjFun/Equality.agda
% src/VSet/Data/InjFun/Properties.agda

\input{latex/generated/DissertationTex/TransformInjFun.tex}
% src/VSet/Transform/InjFun/Compose.agda
% src/VSet/Transform/InjFun/Flatten.agda
% src/VSet/Transform/InjFun/Inflate.agda
% src/VSet/Transform/InjFun/Pred.agda
% src/VSet/Transform/InjFun/Sub.agda
% src/VSet/Transform/InjFun/Tensor.agda
% src/VSet/Transform/InjFun/Properties.agda

\input{latex/generated/DissertationTex/Splice.tex}
% src/VSet/Data/Fin/Splice.agda
% src/VSet/Data/Fin/SumSplit.agda
% src/VSet/Data/Fin/Properties.agda
% src/VSet/Data/Fin/Tests.agda

\input{latex/generated/DissertationTex/Inj.tex}
% src/VSet/Data/Inj/Base.agda
% src/VSet/Data/Inj/InjFun.agda
% src/VSet/Data/Inj/Order.agda
% src/VSet/Data/Inj/Properties.agda
% src/VSet/Data/Inj/Tests.agda

% \input{latex/generated/DissertationTex/TransformInjElementary.tex}
% src/VSet/Transform/Inj/Elementary/Base.agda
% src/VSet/Transform/Inj/Elementary/Extra.agda
% src/VSet/Transform/Inj/Elementary/Properties.agda
% src/VSet/Transform/Inj/Elementary/Tests.agda

% \input{latex/generated/DissertationTex/transformInjCompose.tex}
% src/VSet/Transform/Inj/Compose/Base.agda
% src/VSet/Transform/Inj/Compose/Properties.agda
% src/VSet/Transform/Inj/Compose/Tests.agda

% \input{latex/generated/DissertationTex/transformInjInverse.tex}
% src/VSet/Transform/Inj/Inverse/Base.agda
% src/VSet/Transform/Inj/Inverse/Insert.agda
% src/VSet/Transform/Inj/Inverse/Properties.agda
% src/VSet/Transform/Inj/Inverse/Tests.agda

% \input{latex/generated/DissertationTex/TransformInjTensor.tex}
% src/VSet/Transform/Inj/Tensor/Associator.agda
% src/VSet/Transform/Inj/Tensor/Base.agda
% src/VSet/Transform/Inj/Tensor/Identity.agda
% src/VSet/Transform/Inj/Tensor/Properties.agda

% \input{latex/generated/DissertationTex/TransformInjTrace.tex}
% src/VSet/Transform/Inj/Trace/Base.agda
% src/VSet/Transform/Inj/Trace/Properties.agda
% src/VSet/Transform/Inj/Trace/Tests.agda

% \input{latex/generated/DissertationTex/Cat.tex}
% src/VSet/Cat/Base.agda
% src/VSet/Cat/Monoidal.agda

\chapter{Conclusion}\label{conclusion}

\section{Results}
This dissertation formalizes two category constructions of the
category of finite sets and injective funcitons, providing two
foundations for further development.

\begin{itemize}
  \item We defined two encodings of injective functions:
    \texttt{InjFun}, defined using a dependent sum; and \texttt{Inj},
    defined inductively.
  \item We proved they both form a category
  \item We defined a tensor product for both of these.
  \item We sketched a construction of the monoid laws.
  \item We define a trace operator.
  \item We prove some basic properties (omitted from this report due to time).
\end{itemize}

All definitions were formalized in Cubical Agda, and most of the
lemmas proven were formalized, though there are still significant
holes which require further work to complete.

\section{Discussion}
The central research question (objective \ref{obj:trace-coherence},
and \ref{obj:int-construction}: whether the category \textbf{Inj}
admits a trace satisfying the JSV axioms, which form the foundations
for a full treatment of the \textbf{Int}-construction for compact
closed completionremains only partially answered here. In practice,
the proof burden fell on a long tail of elementary lemmas about finite
sets, sums, and path manipulations. Midway through, I pivoted from the
dependent-sum encoding to an inductive encoding in the hope of
simplifying coherence proofs; however, both approaches incurred
distinct overheads (transport management vs.\ structural bureaucracy),
and neither reached a full monoidal (let alone traced) package within
the project time-frame.

This project turned out to be much more challenging than initially
anticipated. Neither of my approaches gave a simple way to go about
the formalization of the result that injective maps on finite sets
form a traced monoidal category. Progress was limited by the large
amount of lemmas and sub-lemmas that need to be completed. After
feeling like the progress had slowed with the depdent sum half-way
though the project, I thought that starting again with an inductive
approach would be significantly simpler. It turns out this assessment
was wrong and both approaches had significant limitations that I was
unable to overcome to complete the construction of even a symmetric
monidal category let alone a traced monoidal category.

If I was doing this project again, I would shrink the scope
significantly and commit to a single representation for the duration
of the project. This would have been better for either approach if I'd
stuck with it and proven the monoidal axioms before attempting traced
monoidal categories and implementing the \textrm{Int}-construction
\cite{joyal1996-traced-monoidal-categories}.

Additionally I noticed that I hadn't allocated enough time to do the
report justice, and I started it too late. I wasn't able to finish
writing up the formalization of the tensor product on `Inj`, because I
spent too much time trying to make progress on the code, to have
something impressive to show at the end.

The project was very engaging, and I would be very interested in
trying to complete the construction after I have submitted this
dissertation.


\appendix
\chapter{References}
\bibliographystyle{plain}
\bibliography{latex/report}
\chapter{Agda Source Listings}
\end{document}
